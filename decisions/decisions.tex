\documentclass[12pt]{article}
\usepackage[top=0.85in, bottom=0.93in, right=0.93in, left=0.93in,
paperwidth=8.5in, paperheight=11in, nohead]{geometry}
\geometry{letterpaper}
\usepackage[pdftex]{graphicx}
\usepackage{color}
\usepackage[normalem]{ulem}
\usepackage{amssymb}
\usepackage{amsmath}
\usepackage{epstopdf}
\usepackage{setspace}
\usepackage{mdwlist}
\usepackage{verbatim}
\usepackage{tikz}
\usepackage[numbers]{natbib}

%% make bibliography more compact \setlength{\bibsep}{0in}
\renewcommand\bibsection{\subsubsection*{\refname}}

\begin{document}

\begin{centering}
  \large {\bf Pyrodiversity promotes interaction complementarity and
    population resistance: Analysis decision points} \\
\end{centering}
\vspace{0.15in}


\section{Data prep} 
\begin{enumerate}
\item Drop all non-bees. Rational: non-bees not identified to species
  with the same rigor as bees because many groups do not have
  experts. Also, life histories are less well understood, and bees are
  the more dominant pollinator in this system.
\end{enumerate}

\section{Network metrics}
\begin{enumerate}
\item Which metrics best represent redundancy and complementarity?
  There are a plethora of metrics to chose from. I decided on the most
  direct metrics developed primarily by Devoto et al.\
  (https://doi.org/10.1111/j.1461-0248.2012.01740.x) for
  complementarity (previously used for networks), degree (the simplest
  and widely used metric of generalization), and Rao's redundancy
  metric as advised by Ricotta et al.\ 2016 (doi:
  10.1111/2041-210X.12604).
\end{enumerate}

\section{Interaction flexibility}
\subsection{Partner turnover}
\begin{enumerate}
\item Null models for partner variability: Constrain alpha diversity
  and not the number of individuals in the null models for calculating
  the expected $\beta$-diversity. Rational: Little difference in the
  results when different null models are used (see Ponisio et
  al.~2016. Global Change Ecol. https://doi.org/10.1111/gcb.13117)
\item Partner variability calculation: weight interactions by their
  ``abundance'' vs. binary interactions and use Chao as a
  dissimilarity estimate. Rational: turnover in the number of
  interactions is ecological meaningful and a part of interaction
  flexibility. Chao (Chao et al.~2005,
  https://doi.org/10.1111/j.1461-0248.2004.00707.x) is density
  invariant, replication invariance, and monotonic (Jost L, Chao A,
  Chazdon RL. Compositional similarity and $\beta$ diversity In:
  Magurran AE, Mc Gill BJ, editors. Biological diversity: frontiers in
  measurement and assessment. Oxford: Oxford University Press;
  2011. pp. 66–84.)
\item Partner variability calculation: calculate the corrected
  interaction turnover by using the Chase et al.\
  (https://doi.org/10.1890/ES10-00117.1) (calculating the fraction of
  randomly assembled communities with dissimilarity values less than
  and half of those equal to that of the observed
  community). Rational: calculating the corrected turnover using
  z-scores (the alternative) was not qualitatively different (see
  Ponisio et al.~2016. Global Change Ecol.).
\item Species-specific turnover calculation: the CV of the turnover
  across all sampling periods (both through time at a site and across
  sites) in 2013. Rational: 2013 is our baseline, pre-extreme drought
  data point. Interaction flexibility can be both within a season (at
  the same site through time and different plant come into and out of
  bloom) and across sites (re-shuffle the plant community
  composition). Other options, 1) cv corrected by samples size, an
  unbiased estimates when sample sizes are low. Ruled out because
  sample size is not particularly small. The pvalue
  for the interaction between partner variability and pyrodiversity
  becomes marginal (0.06), but this does not substantially change the
  results (unless you are very militant about pvalues). 2) sd. Ruled
  out because does not control for mean. 
\end{enumerate}

\subsection{Role variability}
\begin{enumerate}
\item Role quantification: chose "rarefied degree",
  "weighted.betweenness", "weighted.closeness", "niche.overlap",
  "species.strength", "d" as representatives of a species network
  role. Rational: These metrics cover a range different aspects of a
  species' role, though there is some overlap. Choose different
  similar metrics did not change results qualitatively.
\item Role variability calculation: CV of pca1 scores for each species
  across samples in 2013. Rational: CV measures the variability while
  controlling for the mean of values. SD and corrected CV rules out
  for same reasons as with partner variability. pca1 is the primary
  axis for differentiating roles. Interaction flexibility can be both
  within a season (at the same site through time and different plant
  come into and out of bloom) and across sites (re-shuffle the plant
  community composition).
\end{enumerate}

\subsection{Linear mixed models of drought resistance}
\begin{enumerate}
\item Dropping extrema: I dropped two species with really extreme
  values of role flexibility. Rational: from examining their
  interactions, they seem to have occupied basically the exact same
  role in every network they were observed in (though interestingly
  different species). The model diagnostic look a lot better without
  including them. In addition, the model results are mostly
  consistent. Removing them makes the coefficient for pyrodiversity
  less negative, as expected by removing an extreme value that is
  shaping the slope.
  \end{enumerate}

  
  \subsection{Network resistance to species loss (``robustness'')}
  \begin{enumerate}
  \item What order to drop plant species to simulation a drought-like
    event? Sensible options: from least to most degree or abundance
    calculated from network data or veg survey data. Veg survey data
    is based on the number of flowers vs. individuals, which is a bit
    misleading in terms of drought susceptibility. Dropping species by
    degree (sum of unique bee interaction partners at each site) and
    abundance (sum of total interactions) were qualitatively similar
    in terms on the results of the linear mixed models. Rational: drop
    species by abundance because that is the most like a drought
    perturbation (more likely to loose the least abundant first).
  \end{enumerate}

  
  

\end{document}

%%% Local Variables:
%%% mode: latex
%%% TeX-PDF-mode: t
%%% End:


